%=============================================================================
%   About cellular automaton
%   Reporte de pentominos - Brief Summary
%   Servicio social - Laura Natalia Borbolla Palacios
%=============================================================================

\subsection{Cellular Automaton}

For more information, please consult \cite{cellular-automata-sep,
cellular-automata-kari, cellular-automata-math, j1, j2, j3}.

A cellular automaton is a discrete dynamical system that consists of a set of
simple units (cells) that change their states at each time step, depending of
the states of their neighbors (there are no actions at a distance), according to
a local update rule; the cells evolve in parallel at discrete time steps and
they all use the same update rule.

They are also known as \textbf{CA} and have proof been quite useful, both as
general models of complexity and as specific representations of non-linear
dynamics in a great variety of fields. Despite functioning in a
non-traditionally way, the \textbf{CA} can solve algorithmic problems or
compute functions, all of this with the right suitable rule, of course.

Cellular automata come in a variety of shapes; the type of grid (for example,
square,  triangular or hexagonal cells in a two-dimensional \textbf{CA}) is one
of it's most fundamental properties. The neighborhood over which cells affect
one another is also very important; two common neighborhoods in the case of a
two-dimensional \textbf{CA} on a square grid are the Moore and the von Newmann
neighborhoods.

The \textbf{elementary cellular automata} are the simplest type of cellular
automaton, being a binary, nearest-neighbor, one-dimensional automaton; these
\textbf{CA} have been thoroughly studied. There are 256 \textbf{ECA}.

The best known cellular automaton is the \textit{Game of life}, discovered by
J. H. Conway in 1970; this is a binary automaton with a Moore neighborhood. Over
the years, it has been compiled a library of patterns with various behaviours:

\begin{itemize}
  \item \textbf{Still life}. A fixed point pattern; the update rule keeps each
    cell unchanged.
  \item \textbf{Oscillator}. A temporally periodic pattern; the update rule may
    change the pattern, but after some steps, the original pattern reapears in
    the same location with the same orientation.
  \item \textbf{Spaceship}. A pattern that reappears, after some steps, although
    not necessarily in the same location.
  \item \textbf{Gun}. A pattern that, as an oscillator, periodically returns
    back to the initial state and emits spaceships.
  \item \textbf{Glider gun}. A pattern that emmits gliders.
\end{itemize}

\subsubsection{The Diffusion Rule}
Conway's \textit{Game of Life} inspired the cientific community and soon they
started to experiment with variations of the game's rules to find new and
interesing behaviours, for example, \textit{high life}, \textit{life 43} and
\textit{long life}; later they started experimenting with new neighborhoods,
different shapes in the grids and the numer of dimensions in the universe. One
rule stands out of the rest: the diffusion rule; mainly because it's behaviour
is chaotic and with it, there has been discovered a great deal of gliders,
oscillators, glider guns and puffer trains. The rule is defined as follow:
\begin{itemize}
  \item Any living cell with less than seven neighbours dies out of loneliness.
  \item Any living cell with more than seven neighbours dies suffocated.
  \item Any living cell with seven neighbours lives onto the next generation.
  \item Any death cell with two living neighbours will be born in the next generation.
\end{itemize}
